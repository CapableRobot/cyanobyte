\documentclass[a4paper,12pt,oneside,pdflatex,italian,final,twocolumn]{article}

\usepackage[utf8]{inputenc}
\usepackage{parallel}
\usepackage{siunitx}
\usepackage{booktabs}
\usepackage{fancyhdr}

\usepackage[export]{adjustbox}
\usepackage[margin=0.5in]{geometry}
\addtolength{\topmargin}{0in}

\usepackage{libertine}
\renewcommand*\familydefault{\sfdefault}  %% Only if the base font of the document is to be sans serif
\usepackage[T1]{fontenc}

\title{ BMP280 }
\author{ Nick Felker }
\date{ 2019 }

\begin{document}

\pagestyle{fancy}

\lhead{ Nick Felker }
\chead { 2019 }
\rhead{ BMP280 v0.1.0 }


\onecolumn


\begin{figure}
\begin{minipage}{0.47\textwidth}

\section{Overview}
    Bosch Digital Pressure Sensor
    \begin{itemize}
        \item Device address 119
        \item Address type 7-bit
    \end{itemize}


\end{minipage}
\hfill

\end{figure}


\section{Register Description}
\begin{itemize}
\item Temperature MSB - Part 1 of temperature
\item Temperature LSB - Part 2 of temperature
\item Temperature XLSB - Final part of temperature
\item Digital Temperature Compensation 1 - Used for Celcius conversion
\item Digital Temperature Compensation 2 - Used for Celcius conversion
\item Digital Temperature Compensation 3 - Used for Celcius conversion
\item Pressure MSB - Part 1 of Pressure
\item Pressure LSB - Part 2 of Pressure
\item Pressure XLSB - Part 3 of Pressure
\item Digital Pressure Compensation 1 - Used for Pascals conversion
\item Digital Pressure Compensation 2 - Used for Pascals conversion
\item Digital Pressure Compensation 3 - Used for Pascals conversion
\item Digital Pressure Compensation 4 - Used for Pascals conversion
\item Digital Pressure Compensation 5 - Used for Pascals conversion
\item Digital Pressure Compensation 6 - Used for Pascals conversion
\item Digital Pressure Compensation 7 - Used for Pascals conversion
\item Digital Pressure Compensation 8 - Used for Pascals conversion
\item Digital Pressure Compensation 9 - Used for Pascals conversion
\end{itemize}

\section{Technical specification}
\centering
\begin{tabular}{lcrr}
\toprule
 & Register Name & Register Address & Register Length \\
\midrule
TempMsb & Temperature MSB & 250 & 8 \\
TempLsb & Temperature LSB & 251 & 8 \\
TempXlsb & Temperature XLSB & 252 & 8 \\
DigT1 & Digital Temperature Compensation 1 & 136 & 16 \\
DigT2 & Digital Temperature Compensation 2 & 138 & 16 \\
DigT3 & Digital Temperature Compensation 3 & 140 & 16 \\
PressureMsb & Pressure MSB & 247 & 8 \\
PressureLsb & Pressure LSB & 248 & 8 \\
PressureXlsb & Pressure XLSB & 249 & 8 \\
DigP1 & Digital Pressure Compensation 1 & 142 & 16 \\
DigP2 & Digital Pressure Compensation 2 & 144 & 16 \\
DigP3 & Digital Pressure Compensation 3 & 146 & 16 \\
DigP4 & Digital Pressure Compensation 4 & 148 & 16 \\
DigP5 & Digital Pressure Compensation 5 & 150 & 16 \\
DigP6 & Digital Pressure Compensation 6 & 152 & 16 \\
DigP7 & Digital Pressure Compensation 7 & 154 & 16 \\
DigP8 & Digital Pressure Compensation 8 & 156 & 16 \\
DigP9 & Digital Pressure Compensation 9 & 158 & 16 \\
\bottomrule
\end{tabular}

\raggedright


\section{Functions}

\centering
\begin{tabular}{lc}
\toprule
  & Description \\
\midrule
temperature & Reads the temperature \\
pressure & Reads the atmospheric pressure \\
\bottomrule
\end{tabular}


\raggedright
\subsection{Function temperature }
Reads the temperature \\

\centering
\begin{tabular}{lcr}
\toprule
  & Inputs & Return \\
\midrule
asRaw &
&
output
\\
asCelsius &
&
celsius
\\
\bottomrule
\end{tabular}



\raggedright
\subsection{Function pressure }
Reads the atmospheric pressure \\

\centering
\begin{tabular}{lcr}
\toprule
  & Inputs & Return \\
\midrule
asRaw &
&
output
\\
asHpa &
&
hpa
\\
\bottomrule
\end{tabular}



\raggedright

\end{document}