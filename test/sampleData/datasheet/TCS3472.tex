\documentclass[a4paper,12pt,oneside,pdflatex,italian,final,twocolumn]{article}

\usepackage[utf8]{inputenc}
\usepackage{parallel}
\usepackage{siunitx}
\usepackage{booktabs}
\usepackage{fancyhdr}

\usepackage[export]{adjustbox}
\usepackage[margin=0.5in]{geometry}
\addtolength{\topmargin}{0in}

\usepackage{libertine}
\renewcommand*\familydefault{\sfdefault}  %% Only if the base font of the document is to be sans serif
\usepackage[T1]{fontenc}

\title{ TCS3472 }
\author{ Nick Felker }
\date{ 2019 }

\begin{document}

\pagestyle{fancy}

\lhead{ Nick Felker }
\chead { 2019 }
\rhead{ TCS3472 v0.1.0 }


\onecolumn


\begin{figure}
\begin{minipage}{0.47\textwidth}

\section{Overview}
    Color Light-to-Digital Converter with IR Filter
    \begin{itemize}
        \item Device address 41
        \item Address type 7-bit
    \end{itemize}


\end{minipage}
\hfill

\end{figure}


\section{Register Description}
\begin{itemize}
\item Enable - Enable specific components of the peripheral
\item Clear channel - This is the ambient amount of detected light.
\item Red channel - Red light as an int. Divide by ambient light to get scaled number.
\item Green channel - Green light as an int. Divide by ambient light to get scaled number.
\item Blue channel - Blue light as an int. Divide by ambient light to get scaled number.
\end{itemize}

\section{Technical specification}
\centering
\begin{tabular}{lcrr}
\toprule
 & Register Name & Register Address & Register Length \\
\midrule
enable & Enable & 128 & 8 \\
clear & Clear channel & 180 & 16 \\
red & Red channel & 182 & 16 \\
green & Green channel & 184 & 16 \\
blue & Blue channel & 186 & 16 \\
\bottomrule
\end{tabular}

\raggedright

\section{Fields}

\centering
\begin{tabular}{lcrr}
\toprule
  & Field Name & Register & Bits \\
\midrule
init & Setup the device configuration & enable &
7:0
\\

\bottomrule
\end{tabular}

\raggedright



\end{document}